\documentclass [12pt]{article}
\usepackage{graphicx,amssymb,amsfonts,latexsym,amsmath,amsthm,times}
\usepackage{epsfig}
\usepackage{fancyhdr}
\usepackage{color}
\usepackage[francais, english]{babel}
\setlength{\textwidth}{6.5in} \textheight=8.5in \oddsidemargin 0in \topmargin -0.5 in
\renewcommand{\theequation}{\thesection.\arabic{equation}}

\numberwithin{equation}{section}

%*******************************************************************
% SPECIFY THE PAGE HEADERS THIS WAY: ON THE LEFT: FIRST AUTHOR et al., ON THE RIGHT: SHORT TITLE
%*******************************************************************
\pagestyle{fancy}
\fancyhead{} % clear all header fields
\fancyhead[RO]{\bfseries Titre}
\fancyhead[LO]{\bfseries A. Youssfi et al.}


%*******************************************************************
% NECESSARY TO DEFINE IN ORDER TO GET AN AUTOMATIC NUMBERING OF THEOREMS, DEFINITIONS, ETC.
%*******************************************************************
\newtheorem{theorem}{Theorem}
\newtheorem{acknowledgement}[theorem]{Acknowledgement}
\newtheorem{algorithm}[theorem]{Algorithm}
\newtheorem{axiom}[theorem]{Axiom}
\newtheorem{case}[theorem]{Case}
\newtheorem{claim}[theorem]{Claim}
\newtheorem{conclusion}[theorem]{Conclusion}
\newtheorem{condition}[theorem]{Condition}
\newtheorem{conjecture}[theorem]{Conjecture}
\newtheorem{corollary}[theorem]{Corollary}
\newtheorem{criterion}[theorem]{Criterion}
\newtheorem{definition}[theorem]{Definition}
\newtheorem{example}[theorem]{Example}
\newtheorem{exercise}[theorem]{Exercise}
\newtheorem{lemma}[theorem]{Lemma}
\newtheorem{notation}[theorem]{Notation}
\newtheorem{problem}[theorem]{Problem}
\newtheorem{proposition}[theorem]{Proposition}
\newtheorem{remark}[theorem]{Remark}
\newtheorem{solution}[theorem]{Solution}
\newtheorem{summary}[theorem]{Summary}


%*******************************************************************
%NECESSARY TO OBTAIN THE SECTIONS, SUBSECTIONS, ETC. DEFINED AS 1.1., 2.1.3. FOR INSTANCE
%*******************************************************************
\renewcommand\thesection{\arabic{section}.}
    \renewcommand\thesubsection{\thesection\arabic{subsection}.}
       \renewcommand\thesubsubsection{\thesubsection\arabic{subsubsection}.}
       \renewcommand\theequation{\thesection\arabic{equation}}

\begin{document}

\pagestyle{fancy}
\fancyhf{}
\begin{center}
\lhead{{\bf\hspace*{0.5cm}  	
Les quatri\`{e}mes Journ\'{e}es Internationales d'Analyse Fonctionnelle-Th\'{e}orie Spectrale \\
~~~~~~~~~~~~~~~~~~~~~~~~~~~~~~~~~~~~~~~~~~~~~~~~~~~~~~25-26 octobre 2019,  F\`{e}s, Morocco.
}}
\end{center}
\rhead{}

%*******************************************************************
%TITLE OF THE ARTICLE
%*******************************************************************
\vspace*{2cm} \normalsize \centerline{ \bf On a new fractional Orlicz-Sobolev space and applications to nonlocal variational problems}
\vspace*{1cm}
%*******************************************************************
%AUTHORS - THE CORRESPONDING AUTHOR NEEDS TO SPECIFY HIS/HER E-MAIL ADDRESS AS A FOOTNOTE
%*******************************************************************

\centerline{\bf \underline{Mohammed SRATI} $^{a,}$\footnote{Corresponding author. E-mail: srati93@gmail.com},  Elhoussine AZROUL  $^a$, Abdelmoujib BENKIRANE $^a$}

\vspace*{0.5cm}

%*******************************************************************
% ADDRESS OF THE AUTHORS
%*******************************************************************
\centerline{$^{a,b,c}$  USMBA-FSDM,
 \\ Laboratory of Mathematical Analysis and Applications, Fez, Morocco.}




%*******************************************************************
%ABSTRACT
%*******************************************************************


\vspace*{1cm}

\noindent {\bf Abstract.}
In this talk, we investigate the existence of weak solution for a fractional type problems driven by a nonlocal  operator of elliptic type in a fractional Orlicz-Sobolev space, with homogeneous Dirichlet boundary conditions.
 We first  extend the fractional Sobolev spaces  $W^{s,p}$ to include the general  case $W^sL_A$, where $A$ is an N-function and $s\in (0,1)$. We are concerned with some qualitative properties of the space $W^sL_A$ (completeness, reflexivity and separability). Moreover,  we prove a continuous and compact embedding theorem of these spaces into Lebesgue spaces.


%**********************************************
\vspace*{0.5cm}

%*******************************************************************
%KEYWORDS
%*******************************************************************
\noindent {\bf Key words:} Fractional Orlicz-Sobolev spaces, fractional $a$-Laplace operator, direct method in calculus of variations.
 .\\
%*******************************************************************
%AMS SUBJECT CLASSIFICATION
%*******************************************************************




%*******************************************************************
%BIBLIOGRAPHY
%*******************************************************************
\begin{thebibliography}{99}

  \bibitem{1} R. A. Adams, Sobolev Spaces, Academic Press, New York, 1975.
    
     
    \bibitem{333} V. Ambrosion   Nontrivial solutions for a fractional p-Laplacian problem via Rabier Theorem. Journal of
       Complex Variables and Elliptic Equation, Volume 62, 2017.
       
	 \bibitem{3} E. Azroul, A. Benkirane, M.Srati, Introduction to fractional Orlicz-Sobolev spaces, arXiv:1807.11753 [math.AP] 31 jul 2018.
 	 
  \bibitem{5} Ph. Cl\'ement, M. Garc\'ia-Huidobro, R. Man\'asevich, K. Schmitt Mountain pass type solutions for quasilinear elliptic equations. Calculus of Variations and Partial Differential Equations journal,
 	       August 2000, Volume 11, Issue 1, pp 33-62.
 	     
 	             \bibitem{12} M. A. Krasnosel'skii and Ja. B. Rutickii, Convex functions and Orlicz spaces, Translated from
 	                          the first Russian edition by Leo F. Boron, P. Noordhoff Ltd., Groningen, 1961.
 	     
 
       \bibitem{11}  E. D. Nezza, Giampiero Palatucci, and Enrico Valdinoci, Hitchhiker's guide to the
              fractional Sobolev spaces, Bull. Sci. Math. 136 (2012), no. 5, 521-573. MR 2944369.
        
       \bibitem{9} C. E.T LEDESMA, Existence and symmetry result for Fractional
              	  p-Laplacian in  $\mathbb{R}^N$, Communications on Pure and Applied Analysis . Jan2017, Vol. 16 Issue 1, p99-113. 15p. 
              
\end{thebibliography}
\end{document}
